\chapter{Introduction}\label{c1}

\section{Background}

There is a need to increase the economic effectiveness of wind turbines, which refers to the cost to run them relative to the electricity generation, or revenue \cite{Kim12,Leahy16}. Increasing this effectiveness lowers the payback period of new wind turbines or farms, thus making wind a more economic clean energy source, attracting governments and private organisations to make more investments in wind projects \cite{Kim12}. It can, however, be decreased due to major component failure, frequent downtime, turbine degradation and age, which in turn increase the operation and maintenance cost and decrease the energy generation efficiency of wind turbines \cite{Kim12,Diens16}. There are difficulties and high costs involved in carrying out maintenance on wind turbines, especially for ones that operate in extreme and remote conditions, such as offshore wind farms, where the turbines tend to also exist in larger numbers \cite{Diens16,Tautz17}.

Condition-based monitoring systems that continuously monitor wind turbine states increase this effectiveness by significantly reducing the maintenance costs, reportedly by 20\,\% to 25\.\%, as it prevents unscheduled maintenance \cite{Leahy16}. According to the Electric Power Research Institute, reactive maintenance, which refers to running the turbine until it reaches failure, has the highest cost, followed by preventive or scheduled maintenance, which is reported to cost 24\,\% less \cite{Wind15}. Meanwhile, condition-based or predictive maintenance, which prevents catastrophic failure, \cite{Kim12} is reported to save 47\,\% of the cost of reactive maintenance, \cite{Wind15} which makes it the most cost-effective and preferred approach. Condition-based monitoring technologies include sensor-based oil and vibration analysis, which are useful for checking the oil for properties such as temperature, and rotating equipment respectively \cite{Garci12}. These technologies, however, tend to put emphasis on the more expensive parts of a wind turbine such as the gearbox \cite{Godwi13} due to the high costs involved in the installation of these sensors \cite{Leahy16,Garci12}. These systems, which can be purchased from the turbine manufacturer, are usually pre-installed in offshore wind turbines due to the harsh environments in which they operate. However, they can be expensive \cite{Tautz17} and uneconomical, especially for older wind turbines in onshore wind farms, whose outputs are often less than that of an offshore wind farm.

An alternative would be to use SCADA-based analysis, where the only cost involved would be computational and expensive sensors are not required \cite{Tautz17,Leahy16}. A SCADA system, which stands for supervisory control and data acquisition, found pre-installed in most utility-scale wind turbines, collects data using numerous sensors at the controllers with usually 10-minute resolution \cite{Tautz17,Yang14}, of various parameters of the wind turbine, such as wind speed, active power, bearing temperature and voltage \cite{Leahy16}. Power curve analysis can be done using this data, but this analysis only detects wind turbine underperformance \cite{Gill12}. Meanwhile, implementing machine learning algorithms on SCADA signals to classify them as having either normal or anomalous behaviour, has the ability to predict faults in advance. This has been demonstrated in a number of publications.

Kusiak and Li \cite{Kusia11} investigated predicting a specific fault, which is diverter malfunction. 3 months' worth of SCADA data of four wind turbines were used and the corresponding status and fault codes were integrated into this data to be labelled to differentiate between normal and fault points. To prevent bias in prediction in machine learning, the labelled data is sampled at random, ensuring the number of samples with a fault code is comparable to the number of normal samples. Four classification algorithms, namely neural networks, boosting tree, support vector machines, and classification and regression trees were trained using two-thirds of this data which was randomly selected. The boosting tree, which was found to have the highest accuracy of 70\,\% for predicting specific faults, was investigated further. The accuracy of predicting a specific fault at the time of fault was 70\,\%, which decreased to 49\,\% for predicting it 1 hour in advance. Only one specific fault was the focus of this methodology and in reality, wind turbines could have many faults in different components and structures, which may all have some form of correlation between one other.

Godwin and Matthews \cite{Godwi13} focussed on wind turbine pitch control faults using a classifier called the RIPPER algorithm. They used 28 months' worth of SCADA data containing wind speeds, pitch motor torques and pitch angles, of eight wind turbines known to have had pitch problems in the past. The classes used were normal, potential fault and recognised fault. Using maintenance logs, data up to 48 hours in advance was classed as recognised fault, data in advance of this with corresponding SCADA alarm logs indicating pitch problems was classed as potential fault, and the remaining unclassed data was classed as normal. Random sampling was performed here as well to balance the classes and prevent bias. The data of four turbines were used to train the RIPPER algorithm, and the remaining four used for testing. The analysis was done using the entire 28 months of data as well as 24, 20, 16, 12, 8 and 4 months of data to find out how the amount of data affects the accuracy of classification. Using the entire 28 months of data was found to produce the most accurate classifier, with a mean accuracy of 85\,\%. Looking at the results in more depth, it was found that the classifier had F1 scores, which is an accuracy measure that accounts for true and false positives and negatives, of 79\,\%, 100\,\% and 78\,\% in classifying normal, potential fault and recognised fault data respectively. Although the results are an improvement to Kusiak and Li \cite{Kusia11}, this methodology similarly focussed on only one fault.

Leahy et al. \cite{Leahy16} used a specific fault prediction approach, implementing a support vector machine classifier from scikit-learn's LibSVM. They used SCADA data from a single 3 MW wind turbine spanning 11 months with status and warning codes. The labelling was done such that data with codes corresponding to the turbine in operation, low and storm wind speeds represent normal conditions and codes corresponding to each specific fault to represent faulty conditions. Data preceding these fault points by 10 minutes and 60 minutes were also labelled as faults in separate sets and the effects of using these different time scales to predict faults were investigated. For data identified as normal, filters were applied to remove curtailment and anomalous points. The classifier's hyperparameters were optimised using randomised grid search and validated using ten-fold cross-validation, and the classes were balanced using class weights. Separate binary classifiers were trained to detect each specific type of fault, which were faults in air cooling, excitation, generator heating, feeding and mains failure. The prediction of generator heating faults 10 minutes in advance had the best results, with F1 scores of 71\,\% and 100\,\% using balanced and imbalanced training data respectively. This increase in score using imbalanced data was attributed to the test set having very few instances with the fault class relative to normal data. The same fault, when predicted 60 minutes in advance, had F1 scores of 17\,\% and 100\,\% using imbalanced and balanced training data respectively. Although the score is perfect and it demonstrates the effects of using balanced datasets, the classification again is done separately for each specific fault and it performed poorly on other faults. For instance, detecting excitation faults 10 and 60 minutes in advance using balanced training data only yielded F1 scores of 8\,\% and 27\,\% respectively.

This project will therefore focus on integrating the ability to predict multiple faults at different time scales simultaneously.

\section{Objectives}

The first objective of this project is to implement a classification algorithm on wind turbine SCADA signals to identify underperforming turbines. This involves setting-up the machine learning environment, processing operational data and reporting initial results obtained through implementing a classification algorithm on the data.

The second objective is to create an effective methodology for the integration of failures and to present and interpret results. This includes labelling the data such that each specific fault can be differentiated, evaluating the performances of several classification algorithms to find the most suitable classifier, identifying limitations and suggesting improvements to the method and how it can be adapted for use in industry.

\section{Outline}

\autoref{c2} will describe in detail the tools and datasets used, how the data was processed and labelled and the classification methods and performance metrics used. In \autoref{c3}, a detailed description of the results obtained is presented, followed by a discussion of these results and limitations of this methodology in \autoref{c4}. In \autoref{c5}, conclusions are drawn and possible areas for future work are recommended.

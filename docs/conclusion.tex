\chapter{Conclusions}\label{c5}

A methodology for predicting multiple wind turbine faults in advance by
implementing classification algorithms on wind turbine SCADA signals was
proposed. 30 months' worth of SCADA data for a wind farm with 25 turbines was
processed and labelled using corresponding downtime data containing turbine
categories that describe the condition of the turbine. Since the multiple
faults are treated as separate labels, multiclass-multilabel classification
algorithms, namely DT, RF and kNN, offered in scikit-learn, the machine
learning library for Python, were analysed. In order to predict faults in
advance, three types of classes were used: `normal' to indicate normal
behaviour, `faulty' to indicate a fault, and `X hours before fault' (where X
\ensuremath{=} 6, 12, \dots, 48) to detect faults in advance at varying time scales.
This will allow predictive maintenance to be done appropriate to the time
scale to prevent catastrophic failure to the turbine. Each of these
classifiers have hyperparameters which were tuned for optimal performance on
the data using five-fold cross-validation. The effects of balancing training
data were also investigated.

The use of multilabel-multiclass algorithms allowed for the classification of
each turbine to be done using a single estimator which produces the results of
all labels simultaneously and has a shorter training time. Of the three
classifiers, RF was found to have the best performance overall. A detailed
analysis was done on the results for two labels, namely `electrical system',
which had the worst performance, and `gearbox' which had relatively better
performance. The performance of the `X hours before fault' classes was found
to be relatively poor compared to the other two classes. The performance of
these classes was slightly better using balanced training data. This has
drawbacks, including using separate estimators for each label, and increased
training time and use of resources due to the use of larger training data.
After evaluating feature importance, it was concluded that the poor
performance of these classes could be attributed to the features used or
errors in the data. Further work should be done using additional SCADA fields
relevant to each label to verify this. After additional improvements to the
model and conducting a cost function analysis, the method could be tested in
industry.
